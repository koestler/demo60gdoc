\chapter{Conclusion \& Outlook}
\section{Conclusion}
During the evaluation of different receiver architectures, it was found
that sub-Nyquist sampling significantly reduces the analog receiver complexity.
Using quadrature sub-Nyquist sampling gives even more flexibility in frequency
planning. \\

The chosen approach to build an \gls{FPGA} based hardware platform to get
actual measurements and using Matlab to implement all signal processing steps
in software has proven to be very efficient. The Matlab implementations
allowed for much faster development compared to implementing everything on the
\gls{FPGA}. Thereby more time was left to investigate different designs and
effects. \\

The real hardware in contrast to the simulations allowed to find
different hardware impairments like phase noise, frequency selectivity of
miscellaneous components, amplifier compression and local oscillator leakage at
mixers were found while using the hardware platform. \\

Finally it was shown, that an \gls{EVM} of -30 dB can be achieved using a channel
bandwidth of 450 MHZ, which allows for \gls{QAM}-256 modulation.
When using the full channel width of 1.8 GHz, the non-perfect
reconstruction of the analytic \gls{RX} \gls{IF} signal lead to a reduction of the
\gls{EVM} to -17 dB.

\section{Outlook}
The results suggest that the reconstruction of the analytic can be further
improved by channel estimation and correction of both samples channels
separately. This would allows to correct the different frequency selectivity
of the two outputs of the $90^\circ$ coupler shown in \secref{sec:comp_90deg}. \\

Also the implemented \gls{FPGA} design provides a good platform
for further circuit development. One could immediately start to
move receiver components, which are currently implemented in Matlab,
to programmable logic since the tricky interfaces to the \gls{ADC}
and the \gls{USB} communication with a host computer are already implemented.

%%  LocalWords:  Multi QAM Matlab FPGA Nyquist EVM coupler
