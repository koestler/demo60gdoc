\chapter{Communication and Measurement System}
In order to analyse the different receiver designs described in
\chapref{chap:rx} and to experiment with different channels
and system parameters a versatile comminication system
including a simulation framework and interfaces to the hardware
was needed. \\

The primary aim of the setup to show, that multi-gigabit per second
throughput is possible using high modulation and wide channels.
Also the effect of phase noise and other performance limiting impairments
were investigated. \\

To achiev all this goals a flexible Matlab script supporting
many different configurations was written. It allows to simulate
the different receivers designs but also interfaces to the hardware
to replace the \gls{RF} parts by real hardware. Matlab was chosen
because of it's widespread use in communication systems research. \\

To optimally contribute to existing work, many parameters including
the important parts of the frame structure as well as modulation
schemes were done according to the proposed IEEE 802.11ad standard.
Since this system is soley for research and to avoid spending any
time on unnecessary features, the used system is not compliant to the
proposed standard though. \\

\section{Transmitter}
Since the implementation of the \gls{RF} part of the
transmitter was of minor intreset for this work,
an \gls{AWG} described in \secref{osec:comp_awg}
and a Sivers 58-63 GHz converter were used. \\

\section{Modulation and Pulse Shaping Scheme}
Because the training and estimation fields are not used to 
transmitt any information but only to synchronize in time
and frequency and to measure channel properties they were all
modulated using \gls{BPSK}. \\

For the data fields \gls{BPSK}, \gls{QPSK} and arbitrary, even
\gls{QAM} modulations were implemented. Therby the datarate
can be maximized depending on the \gls{SNR}. \\

A \gls{RRC}-filter was used for pulse shaping. It has the big advantage that
the same filter can be used for transmission and reception. \\

\section{Frame strucutre}
\label{sec:sys_frame_struct}

Very similar to the standard, the transmitted frames are build as shown in
\figref{fig:sys_frame_struct}.



\begin{itemize}
\item 802.11ad oriented
\item ZEROS FES CES FIRST\_PES DATA PES DATA PES ... ZEROS
\item Explain reason of all fields
\item Explain cyclic properties
\end{itemize}

\section{Frequency offset estimation and correction}

\section{Phase noise estimation and correction}
\begin{itemize}
\item Cite phase noise measurements of Radoslav
\item Explain how to estimate and correct it
\item reference to own results in later chapter
\end{itemize}

\section{Simulation}
\begin{itemize}
\item General simulation flow using Matlab
\item Simulated scenarios:
  \begin{itemize}
  \item Baseband receiver
  \item Simulated 90 deg coupler using hilbert transform
  \item hardware baseband QI receiver
  \item hardware intermediate frequency QI receiver
  \end{itemize}
\item Interface to AWG
\item Interface to Oscilloscope
\item Interface to FPGA
\item Replace more and more by hardware
\end{itemize}

