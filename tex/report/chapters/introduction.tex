\chapter{Introduction}
\label{ch:introduction}

\section{Importance of continuous Development in
  Mobile Communication Technologies}

Mobile communication systems became so ubiquitous that most of our daily
use of information technology simply would not work without them.
In many first world countries, one has to almost search for places without
mobile cell network coverage and in third world countries the availability of
mobile communication is often better then the one of fresh water. \\

Because of their high availability, computer networks became so closely
incorporated in many applications, that their performance highly depend
on a low latency and high bandwidth link to a server providing data or
even doing the actual task. Therefor it is of essential importance that
the performance of computer networks continuous \cite{web_content_delivery}
to grow exponentially at least as fast as other key factors like
\gls{CPU} speeds and memory capacities. \\

\acrshort{CMOS} technology and thereby digital signal processing capabilities
develops according to Moore's Law, while this is very unclear for analog
circuits in near future as stated in \cite{belgium}.
Therefor a trend to move processing steps from analog to digital
domain exists. \\

\section{Role of 60 GHz}
Since it won't be possible to increase the throughput of digital communication
systems forever by improving the \gls{SNR} as well as coding, channel allocation
and modulation schemes, the only way will be to increase the channel bandwidth.
While \gls{VHF} and \gls{UHF} bands are heavily used by many different
applications and because implementation constraints require the channel
bandwidth to be small compared to the absolute carrier frequency,
the only way to much higher carrier frequencies. \\

Therefor this work experiments with the possibilities of the 60 GHz
\gls{ISM} band where the available setup allows for bandwidth of up to
1.8 GHz and therefor data rates up to 7.2 Gbit/s. \\

Such high data rats allow for new applications which simply were not possible
using existing wireless standards. Network storage devices can be accessed
at close distance with their full hard disk speed to allow for fast file
up- and download. \gls{HD} video can be streamed lossless and uncompressed
over the air possibly replacing existing cable connections to video
projectors. \\

60 GHz confronts engineers with many new design constraints though. The very
high free-space path loss requires for advanced beam forming methods
for distances above a few meters.
This is possible, since the low wave length allow to integrate
big antenna arrays directly on an \gls{IC}. \\

On the other hand, the high attenuation allows to reuse the same channel
in a relatively close cell and at the current time results in an almost
always completely silent band. \\

While the object sizes and distances in an indoor environment stay the same,
the symbol rates on 60 GHz get much higher. Therefor the resulting
delay spread is wider and correct channel estimation and correction
gains importance.  \\

\section{This Thesis}
In order to analyze the different receiver designs,
to get measurements of channel characteristics,
to quantify phase noise figures and finally to find
performance limiting impairments on 60 GHz communication
systems, a versatile communication system was needed. \\

Therefor different receiver architectures were analyzed as described in
\chapref{chap:rx} and a communication system including time and frequency
synchronization, channel and phase noise estimation and correction was
implemented in software as described in \chapref{chap:sys}.
\chapref{chap:fpga} described the \gls{FPGA} design, which was developed
to interface a very high speed 3.6 GS/s \gls{ADC},
store it's data in real time and provide a fast download interface
and thereby allows to reproduce the simulation results on real hardware.
All the used \gls{RF} components were measured using a state of the art
vector network analyzer as noted in \chapref{chap:comp}.
Finally \chapref{chap:res_450} and \chapref{chap:res_1800} contain information
about the exact test setup and the achieved performance of the system. \\

%%  LocalWords:  interferer rx CMOS SNR Gbit HD lossless IC sys fpga
%%  LocalWords:  FPGA
