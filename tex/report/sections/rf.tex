\chapter{Receiver Design}
\label{chap:rx}
\section{Receiver Architecture}
\subsection{Quadrature Baseband Sampling Receiver}

\begin{itemize}
\item Band filter: very difficult to make ajastuble -> filter whole band
\item First IF has to be high enough such that the undesired side band lies
  outside the band filter.
\item high enough is > rf bandwidth / 2 (when USB / LSB  can be switched later)
\item since $\omega_{\text{LO}_1}$ for the 60 GHz case is high enough, the mirror
  of the desired signal at $f_{\text{RF}} + \omega_{\text{LO}_1}$ is filtered
  by the implementation of the filter.
\item channel filter: suppressed signal from nabourigh channels
\item use quadrature sampling because of assymetric signal
\item Low-Pass filter to avoid aliasing
\item Con: Self-Missing of $\omega_{\text{LO}_2}$ leads to DC-Offset that has to be blocked using a
  High-Pass-Filter also removes a small part of the signal
\item erste Mixer -> hoch, LSB aus Band
\item zweiter Mixer -> tief, besserer Mixer, wenig Carrier leakage
\end{itemize}

\begin{figure}[ht]
  \centering
  \includegraphics[width=\textwidth]{figures/quad_base_rx_block_diagram}
  \caption{Block Diagram of Quadrature Baseband Sampling Receiver}
  \label{fig:rx_quad_base_bd}
\end{figure}

\subsection{Intermediate Frequency Sampling Receiver}
\begin{itemize}
\item Two mixer paths and $90^\circ$ couplers can be used to optain a
  single sided donmixed signal
\item High-Pass filter removes signal between RX $\text{LO}_1$ and channel (including TX LO)
\item Second Mixer mixes to first Nyquist zone, Low-Pass filter avoids aliasing
\item RX and TX-LO offset to make sure sent LSB residual does not possitively
  interfer with signal by non perfect LSB suppression in RX.
\item Pro: No high first IF to suppress LSB
\item Con: Sharp high- / low-pass Filter needed
\item Con: high ADC sampling speed
\item Presentation: show theory of 90 deg cancelation
\end{itemize}

\begin{figure}[ht]
  \centering
  \includegraphics[width=\textwidth]{figures/if_rx_block_diagram}
  \caption{Block Diagram of Intermediate Frequency Sampling Receiver}
  \label{fig:rx_if_bd}
\end{figure}

\begin{figure}[ht]
  \begin{subfigure}{\textwidth}
    \centering
    \includegraphics[width=0.8\textwidth]{figures/if_rx_freq_tx_if}
    \caption{\gls{TX} \gls{IF}}
    \label{fig:rx_if_freq_tx_if}
  \end{subfigure}
  \vspace{8ex} \\
  \begin{subfigure}{\textwidth}
    \centering
    \includegraphics[width=0.8\textwidth]{figures/if_rx_freq_rf}
    \caption{\gls{RF}}
    \label{fig:rx_if_freq_rf}
  \end{subfigure}
  \vspace{8ex} \\
  \begin{subfigure}{\textwidth}
    \centering
    \includegraphics[width=0.8\textwidth]{figures/if_rx_freq_rx_if1}
    \caption{\gls{RX} high \gls{IF}}
    \label{fig:rx_if_freq_rx_if1}
  \end{subfigure}
  \vspace{8ex} \\
  \begin{subfigure}{\textwidth}
    \centering
    \includegraphics[width=0.8\textwidth]{figures/if_rx_freq_rx_if2}
    \caption{\gls{RX} low \gls{IF}}
    \label{fig:rx_if_freq_rx_if2}
  \end{subfigure}
  \caption{Intermediate Frequency Sampling Receiver in Frequency Domain}
  \label{fig:rx_if_freq}
\end{figure}

\subsection{Quadrate Intermediate Frequency Sub-Nyquist Sampling Receiver}
\begin{itemize}
\item USB / LSB problem similar to IF-Mixer image suppression
\item Problem can be solved by using only one mixer and quadrature sampling.
\item Needs double the bandwidth to sample both channels with the full
  signal bandwidth
\item Use sub-nyquist sampling to avoid DC-offset problem and low-frequency
  limitation of components (carrier leakage)
\item Only very few components needed -> less non-idealities
\end{itemize}

\begin{figure}[ht]
  \centering
  \includegraphics[width=\textwidth]{figures/quad_if_rx_block_diagram}
  \caption{Block Diagram of Quadrate Intermediate Frequency Sub-Nyquist Sampling Receiver}
  \label{fig:rx_quad_if_bd}
\end{figure}

\section{Generation of Analytic Signal In Analog Domain}
\begin{itemize}
\item Explain how perfect analytic signal is created using hilbert transform
\item Explain how it is created using the sin/cos mixer
\item Explain error introduced in non perfect case
\end{itemize}
