\chapter{FPGA}
A powerfull Virtex-7 \gls{FPGA} was used to for the real time signal processing.
During this thesis the data from the \gls{ADC} is acquired and stored at in real
time and than passed to Matlab running on a personal computer to both test
the analog part of the system while keeping the fast implementation speed
and flexibility of Matlab to test the rest of the digital signal processing steps
of the receiver. Nevertheless the architecture was choosen such that it is easy
to step by step replace parts of the signal processing from Matlab to the
\gls{FPGA}.

\section{Architecutre Overview}
The three main tasks of the digital design are to acquire and store the data
produced by the \gls{ADC} in real time and to download this data to a
personal computer in a reasonable time for further processing. \\

As shown in \figref{fig:fpga_architecture_overview} the most important
modules of the design are the \verb|Adc12d1800| which interfaces the
\gls{ADC}, \verb|Ram| which uses a DDR3-\gls{RAM} to store the data
and \verb|Microblaze|, a soft-core processor, which controls the
\gls{USB} 2 data download. \verb|Data2Ram| is interface logic used
to connect the \gls{Ram} and \gls{Adc} and \verb|AxiSlave| implements
an interface between the Axi-Bus used by the Microblaze processor
and the custom \verb|Ram| module. \\

In the following sections each of this modules is shortly described followed
by some insights how clocking and reset is implemented.

\begin{figure}[ht]
  \centering
  %\includegraphics[width=\textwidth]{figures/}
  \caption{Overview of the architecture}
  \label{fig:fpga_architecture_overview}
\end{figure}

\begin{itemize}
\item Some words about where more receiver functions can be added
\item Req/Ack-protocoll, everything stallable
\end{itemize}

\subsection{Req/Ack Protocol}
\label{sec:fpga_reqack}
Between all modules a simple comminucation protocoll is used that allows
one way communication whenever the sender and the receiver are both
ready. As a consequence all module have to be stallable. By only using
one such protocoll arbitrary modules can often be connected
without any additional glue logic. \\

The protocol requires the sender to provide a request output which
is asserted if it has data to send. The receiver only than can apply
an assert output if its acknowledge output if it is ready to receive
the data in at the next positive clock edge. Data is transferred whenever
there is a positive clock edge and req and ack are both asserted (high). \\

Because the acknowledge output of a receiver by definition always depends on the
request input of a receiver, this leads to long chains of routing and
\gls{LUT} resources if not implemented well. Therefor a simple so called
req/ack breaker (see \figref{fig:fpa_reqack_breaker} is used to seperate
this chains by a register.
This has also the advantage that long routing distances can be seperated by inserting
registers into the data and control path.

\todo{add reference to req/ack paper}

\begin{figure}[ht]
  \centering
  %\includegraphics[width=\textwidth]{figures/}
  \caption{Design of the Req/Ack Breaker}
  \label{fig:fpga_reaack_breaker}
\end{figure}

\subsection{Data Acquisition: Adc12d1800}
Data acquistions is done by the module \verb|Adc12d1800| which interfaces
the \verb|Adc12d1800rf| by Texas Instruments. In our case the ADC is configured
to sample two channels, an in-phase and a quadrature phase channel, at up to 1.8 GHz
or to sample one channel at 3.6 Ghz. The nominal resolution is 12 bits.
This results in a total bandwidth of $5.4 \text{Gb}/\text{s}$.
Since 1.8 GHz and especially 3.6 GHz is far beyond what the IO-Pads of all
\glspl{FPGA} \todo{reference needed} and most \glspl{ASIC} can handle,
the ADC can output on 4 parallel streams of 12 bit each. \\

Since this 4 streams might be routed with different wire lengths on a \gls{PCB}
each stream has it's clock aligned to the data.
In order to half the maximum frequency of the clock line to the same as the one
of the data lines, the data change on positive and negativ clock edges
(\gls{DDR}). \\

At full speed, this results in a switching frequency of 450 MHz for all data
and clock lines. To support this high speed while reducing power consumption,
and the influence of noise compared to classical \gls{CMOS} signaling, 
\gls{LVDS} is used. \\

Since the maximum clock frequencies of the Virtex \glspl{FPGA} for logic
including \gls{LUT} slices and routings with fanouts above 4
are limited to around 200 to 300 MHz, the input stream needs to be further
parallelized. \\

As shown in \figref{fig:fpga_adc} the implemented design uses
a total of 5 stages to paralellize and concentrate the data to only
one single data rate stream following the req/ack protocol described in
\secref{sec:fpga_reqack}.

\subsubsection{Stage 0: Differential to Single-ended Conversion}
Each of the 12 bits and the clock of the 4 streams arrive at at the input bank
of the \gls{FPGA} at two neighbouring pads forming a \gls{LVDS} pair.
First this differential signal is converted to a single ended signal using
{\emDifferential Signaling Input Buffer} (IBUFDS) primitives located next
to the pads. This results in $4 \times 12$ bits single ended double data rates
signals plus 4 single ended clock signals at up to 450 MHz.

\subsubsection{Stage 1: Double Data Rate to Single Data Rate Conversion}



\begin{itemize}
\item Challenges of high speed data aquisition
\item FMC Plug
\item 4 x 12 bits ddr at 450 MHz
\item IDDR, INFIFO placement
\item Clock distribution
\item difficulties with reset 
\end{itemize}

\subsection{Storage}
\begin{itemize}
\item DDR3 and its difficulties (refresh, address scheme etc.)
\item Bandwidth calculcation (mes. numbers of datathroughput)
\item MIG of xilinx and it's userinterface
\item pitfall: Mask-Bits 
\end{itemize}

\subsection{Download}
\begin{itemize}
\item considered options: UART, Ethernet, USB2
\item Used USB-PHY and Usb2 Ip Core of xilinx
\item Microblaze software and its features
\item Implemented endpoints and the transfer protocol
\item Linux software for download

\section{Clock domains}
\section{Resets}

\end{itemize}
